% !TEX encoding = UTF-8 Unicode
%!TEX root = thesis.tex
% !TEX spellcheck = en-US
%%=========================================
\chapter{Existing Solutions}


\section{CyberSea Simulator}
The CyberSea Simulator developed by Marine Cybernetics is a simulator for HIL testing of Dynamic Positioning (DP) systems.

Key points from [\cite{HILtestingDP}]:
\begin{itemize}
\item Capabilities for data logging and real-time presentation of results
\item Emphasis on vessel dynamics and accurate simulation of vessel motion at low speed ( < 3kts, wave, wind and current loads (of course, because of DP) in six degrees of freedom "using a nonlinear rigid-body model of the vessel".
\item Several options for interface between HIL Simulator and Computer Control System ("Analog, digital, serial/NMEA protocol", normal network protocol or "dedicated test I/O built into the DP computer system").
\item Generation of realistic signals from all the common sensors and position reference systems (such as "Gyro-compasses, VRUs, wind sensors, thruster feedback [...], power feedback from thrusters, switchboard and generator sets") used in modern DP technology "contaminated with typical noise levels".
\item Advanced generation of GNSS signals with possibility of simulating a broad specter of common failure modes.
\end{itemize}


\section{MSS (Fossen)}

\section{MCSim (Marine Cybernetics)}

\section{Gazebo (ROS)}